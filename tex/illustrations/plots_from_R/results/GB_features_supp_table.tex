% latex table generated in R 4.3.0 by xtable 1.8-4 package
% Sat Jun 10 22:23:05 2023
\begin{longtable}{p{3cm}p{12cm}}
  \toprule
Feature_ID & Name \\ 
  \midrule
GB024a & Is the order of the numeral and noun Num-N? \\ 
  GB024b & Is the order of the numeral and noun N-Num? \\ 
  GB025a & Is the order of the adnominal demonstrative and noun Dem-N? \\ 
  GB025b & Is the order of the adnominal demonstrative and noun N-Dem? \\ 
  GB065a & Is the pragmatically unmarked order of adnominal possessor noun and possessed noun PSR-PSD? \\ 
  GB065b & Is the pragmatically unmarked order of adnominal possessor noun and possessed noun PSD-PSR? \\ 
  GB130a & Is the pragmatically unmarked order of S and V in intransitive clauses S-V? \\ 
  GB130b & Is the pragmatically unmarked order of S and V in intransitive clauses V-S? \\ 
  GB193a & Is the order of the adnominal property word (ANM) and noun ANM-N? \\ 
  GB193b & Is the order of the adnominal property word (ANM) and noun N-ANM? \\ 
  GB203a & Is the order of the adnominal collective universal quantifier (UQ) and noun UQ-N? \\ 
  GB203b & Is the order of the adnominal collective universal quantifier (UQ) and noun N-QU? \\ 
  GB020 & Are there definite or specific articles? \\ 
  GB021 & Do indefinite nominals commonly have indefinite articles? \\ 
  GB022 & Are there prenominal articles? \\ 
  GB023 & Are there postnominal articles? \\ 
  GB026 & Can adnominal property words occur discontinuously? \\ 
  GB027 & Are nominal conjunction and comitative expressed by different elements? \\ 
  GB028 & Is there a distinction between inclusive and exclusive? \\ 
  GB030 & Is there a gender distinction in independent 3rd person pronouns? \\ 
  GB031 & Is there a dual or unit augmented form (in addition to plural or augmented) for all person categories in the pronoun system? \\ 
  GB035 & Are there three or more distance contrasts in demonstratives? \\ 
  GB036 & Do demonstratives show an elevation distinction? \\ 
  GB037 & Do demonstratives show a visible-nonvisible distinction? \\ 
  GB038 & Are there demonstrative classifiers? \\ 
  GB039 & Is there nonphonological allomorphy of noun number markers? \\ 
  GB041 & Are there several nouns (more than three) which are suppletive for number? \\ 
  GB042 & Is there productive overt morphological singular marking on nouns? \\ 
  GB043 & Is there productive morphological dual marking on nouns? \\ 
  GB044 & Is there productive morphological plural marking on nouns? \\ 
  GB046 & Is there an associative plural marker for nouns? \\ 
  GB047 & Is there a productive morphological pattern for deriving an action/state noun from a verb? \\ 
  GB048 & Is there a productive morphological pattern for deriving an agent noun from a verb? \\ 
  GB049 & Is there a productive morphological pattern for deriving an object noun from a verb? \\ 
  GB051 & Is there a gender/noun class system where sex is a factor in class assignment? \\ 
  GB052 & Is there a gender/noun class system where shape is a factor in class assignment? \\ 
  GB053 & Is there a gender/noun class system where animacy is a factor in class assignment? \\ 
  GB054 & Is there a gender/noun class system where plant status is a factor in class assignment? \\ 
  GB057 & Are there numeral classifiers? \\ 
  GB058 & Are there possessive classifiers? \\ 
  GB059 & Is the adnominal possessive construction different for alienable and inalienable nouns? \\ 
  GB068 & Do core adjectives (defined semantically as property concepts such as value, shape, age, dimension) act like verbs in predicative position? \\ 
  GB069 & Do core adjectives (defined semantically as property concepts; value, shape, age, dimension) used attributively require the same morphological treatment as verbs? \\ 
  GB070 & Are there morphological cases for non-pronominal core arguments (i.e. S/A/P)? \\ 
  GB071 & Are there morphological cases for pronominal core arguments (i.e. S/A/P)? \\ 
  GB072 & Are there morphological cases for oblique non-pronominal NPs (i.e. not S/A/P)? \\ 
  GB073 & Are there morphological cases for independent oblique personal pronominal arguments (i.e. not S/A/P)? \\ 
  GB074 & Are there prepositions? \\ 
  GB075 & Are there postpositions? \\ 
  GB079 & Do verbs have prefixes/proclitics, other than those that only mark A, S or P (do include portmanteau: A \& S + TAM)? \\ 
  GB080 & Do verbs have suffixes/enclitics, other than those that only mark A, S or P (do include portmanteau: A \& S + TAM)? \\ 
  GB081 & Is there productive infixation in verbs? \\ 
  GB082 & Is there overt morphological marking of present tense on verbs? \\ 
  GB083 & Is there overt morphological marking on the verb dedicated to past tense? \\ 
  GB084 & Is there overt morphological marking on the verb dedicated to future tense? \\ 
  GB086 & Is a morphological distinction between perfective and imperfective aspect available on verbs? \\ 
  GB089 & Can the S argument be indexed by a suffix/enclitic on the verb in the simple main clause? \\ 
  GB090 & Can the S argument be indexed by a prefix/proclitic on the verb in the simple main clause? \\ 
  GB091 & Can the A argument be indexed by a suffix/enclitic on the verb in the simple main clause? \\ 
  GB092 & Can the A argument be indexed by a prefix/proclitic on the verb in the simple main clause? \\ 
  GB093 & Can the P argument be indexed by a suffix/enclitic on the verb in the simple main clause? \\ 
  GB094 & Can the P argument be indexed by a prefix/proclitic on the verb in the simple main clause? \\ 
  GB095 & Are variations in marking strategies of core participants based on TAM distinctions? \\ 
  GB096 & Are variations in marking strategies of core participants based on verb classes? \\ 
  GB098 & Are variations in marking strategies of core participants based on person distinctions? \\ 
  GB099 & Can verb stems alter according to the person of a core participant? \\ 
  GB103 & Is there a benefactive applicative marker on the verb (including indexing)? \\ 
  GB104 & Is there an instrumental applicative marker on the verb (including indexing)? \\ 
  GB105 & Can the recipient in a ditransitive construction be marked like the monotransitive patient? \\ 
  GB107 & Can standard negation be marked by an affix, clitic or modification of the verb? \\ 
  GB108 & Is there directional or locative morphological marking on verbs? \\ 
  GB109 & Is there verb suppletion for participant number? \\ 
  GB110 & Is there verb suppletion for tense or aspect? \\ 
  GB111 & Are there conjugation classes? \\ 
  GB113 & Are there verbal affixes or clitics that turn intransitive verbs into transitive ones? \\ 
  GB114 & Is there a phonologically bound reflexive marker on the verb? \\ 
  GB115 & Is there a phonologically bound reciprocal marker on the verb? \\ 
  GB116 & Do verbs classify the shape, size or consistency of absolutive arguments by means of incorporated nouns, verbal affixes or suppletive verb stems? \\ 
  GB117 & Is there a copula for predicate nominals? \\ 
  GB118 & Are there serial verb constructions? \\ 
  GB119 & Can mood be marked by an inflecting word (""""""""auxiliary verb"""""""")? \\ 
  GB120 & Can aspect be marked by an inflecting word (""""""""auxiliary verb"""""""")? \\ 
  GB121 & Can tense be marked by an inflecting word (""""""""auxiliary verb"""""""")? \\ 
  GB122 & Is verb compounding a regular process? \\ 
  GB123 & Are there verb-adjunct (aka light-verb) constructions? \\ 
  GB124 & Is incorporation of nouns into verbs a productive intransitivizing process? \\ 
  GB126 & Is there an existential verb? \\ 
  GB127 & Are different posture verbs used obligatorily depending on an inanimate locatum's shape or position (e.g. 'to lie' vs. 'to stand')? \\ 
  GB129 & Is there a notably small number, i.e. about 100 or less, of verb roots in the language? \\ 
  GB131 & Is a pragmatically unmarked constituent order verb-initial for transitive clauses? \\ 
  GB132 & Is a pragmatically unmarked constituent order verb-medial for transitive clauses? \\ 
  GB133 & Is a pragmatically unmarked constituent order verb-final for transitive clauses? \\ 
  GB134 & Is the order of constituents the same in main and subordinate clauses? \\ 
  GB135 & Do clausal objects usually occur in the same position as nominal objects? \\ 
  GB136 & Is the order of core argument (i.e. S/A/P) constituents fixed? \\ 
  GB137 & Can standard negation be marked clause-finally? \\ 
  GB138 & Can standard negation be marked clause-initially? \\ 
  GB139 & Is there a difference between imperative (prohibitive) and declarative negation constructions? \\ 
  GB140 & Is verbal predication marked by the same negator as all of the following types of predication: locational, existential and nominal? \\ 
  GB146 & Is there a morpho-syntactic distinction between predicates expressing controlled versus uncontrolled events or states? \\ 
  GB147 & Is there a morphological passive marked on the lexical verb? \\ 
  GB148 & Is there a morphological antipassive marked on the lexical verb? \\ 
  GB149 & Is there a morphologically marked inverse on verbs? \\ 
  GB150 & Is there clause chaining? \\ 
  GB151 & Is there an overt verb marker dedicated to signalling coreference or noncoreference between the subject of one clause and an argument of an adjacent clause (""""""""switch reference"""""""")? \\ 
  GB152 & Is there a morphologically marked distinction between simultaneous and sequential clauses? \\ 
  GB155 & Are causatives formed by affixes or clitics on verbs? \\ 
  GB156 & Is there a causative construction involving an element that is unmistakably grammaticalized from a verb for 'to say'? \\ 
  GB158 & Are verbs reduplicated? \\ 
  GB159 & Are nouns reduplicated? \\ 
  GB160 & Are elements apart from verbs or nouns reduplicated? \\ 
  GB165 & Is there productive morphological trial marking on nouns? \\ 
  GB166 & Is there productive morphological paucal marking on nouns? \\ 
  GB167 & Is there a logophoric pronoun? \\ 
  GB170 & Can an adnominal property word agree with the noun in gender/noun class? \\ 
  GB171 & Can an adnominal demonstrative agree with the noun in gender/noun class? \\ 
  GB172 & Can an article agree with the noun in gender/noun class? \\ 
  GB177 & Can the verb carry a marker of animacy of argument, unrelated to any gender/noun class of the argument visible in the NP domain? \\ 
  GB184 & Can an adnominal property word agree with the noun in number? \\ 
  GB185 & Can an adnominal demonstrative agree with the noun in number? \\ 
  GB186 & Can an article agree with the noun in number? \\ 
  GB187 & Is there any productive diminutive marking on the noun (exclude marking by system of nominal classification only)? \\ 
  GB188 & Is there any productive augmentative marking on the noun (exclude marking by system of nominal classification only)? \\ 
  GB192 & Is there a gender system where a noun's phonological properties are a factor in class assignment? \\ 
  GB196 & Is there a male/female distinction in 2nd person independent pronouns? \\ 
  GB197 & Is there a male/female distinction in 1st person independent pronouns? \\ 
  GB198 & Can an adnominal numeral agree with the noun in gender/noun class? \\ 
  GB204 & Do collective ('all') and distributive ('every') universal quantifiers differ in their forms or their syntactic positions? \\ 
  GB250 & Can predicative possession be expressed with a transitive 'habeo' verb? \\ 
  GB252 & Can predicative possession be expressed with an S-like possessum and a locative-coded possessor? \\ 
  GB253 & Can predicative possession be expressed with an S-like possessum and a dative-coded possessor? \\ 
  GB254 & Can predicative possession be expressed with an S-like possessum and a possessor that is coded like an adnominal possessor? \\ 
  GB256 & Can predicative possession be expressed with an S-like possessor and a possessum that is coded like a comitative argument? \\ 
  GB257 & Can polar interrogation be marked by intonation only? \\ 
  GB260 & Can polar interrogation be indicated by a special word order? \\ 
  GB262 & Is there a clause-initial polar interrogative particle? \\ 
  GB263 & Is there a clause-final polar interrogative particle? \\ 
  GB264 & Is there a polar interrogative particle that most commonly occurs neither clause-initially nor clause-finally? \\ 
  GB265 & Is there a comparative construction that includes a form that elsewhere means 'surpass, exceed'? \\ 
  GB266 & Is there a comparative construction that employs a marker of the standard which elsewhere has a locational meaning? \\ 
  GB270 & Can comparatives be expressed using two conjoined clauses? \\ 
  GB273 & Is there a comparative construction with a standard marker that elsewhere has neither a locational meaning nor a 'surpass/exceed' meaning? \\ 
  GB275 & Is there a bound comparative degree marker on the property word in a comparative construction? \\ 
  GB276 & Is there a non-bound comparative degree marker modifying the property word in a comparative construction? \\ 
  GB285 & Can polar interrogation be marked by a question particle and verbal morphology? \\ 
  GB286 & Can polar interrogation be indicated by overt verbal morphology only? \\ 
  GB291 & Can polar interrogation be marked by tone? \\ 
  GB296 & Is there a phonologically or morphosyntactically definable class of ideophones that includes ideophones depicting imagery beyond sound? \\ 
  GB297 & Can polar interrogation be indicated by a V-not-V construction? \\ 
  GB298 & Can standard negation be marked by an inflecting word (""""""""auxiliary verb"""""""")? \\ 
  GB299 & Can standard negation be marked by a non-inflecting word (""""""""auxiliary particle"""""""")? \\ 
  GB300 & Does the verb for 'give' have suppletive verb forms? \\ 
  GB301 & Is there an inclusory construction? \\ 
  GB302 & Is there a phonologically free passive marker (""""""""particle"""""""" or """"""""auxiliary"""""""")? \\ 
  GB303 & Is there a phonologically free antipassive marker (""""""""particle"""""""" or """"""""auxiliary"""""""")? \\ 
  GB304 & Can the agent be expressed overtly in a passive clause? \\ 
  GB305 & Is there a phonologically independent reflexive pronoun? \\ 
  GB306 & Is there a phonologically independent non-bipartite reciprocal pronoun? \\ 
  GB309 & Are there multiple past or multiple future tenses, distinguishing distance from Time of Reference? \\ 
  GB312 & Is there overt morphological marking on the verb dedicated to mood? \\ 
  GB313 & Are there special adnominal possessive pronouns that are not formed by an otherwise regular process? \\ 
  GB314 & Can augmentative meaning be expressed productively by a shift of gender/noun class? \\ 
  GB315 & Can diminutive meaning be expressed productively by a shift of gender/noun class? \\ 
  GB316 & Is singular number regularly marked in the noun phrase by a dedicated phonologically free element? \\ 
  GB317 & Is dual number regularly marked in the noun phrase by a dedicated phonologically free element? \\ 
  GB318 & Is plural number regularly marked in the noun phrase by a dedicated phonologically free element? \\ 
  GB319 & Is trial number regularly marked in the noun phrase by a dedicated phonologically free element? \\ 
  GB320 & Is paucal number regularly marked in the noun phrase by a dedicated phonologically free element? \\ 
  GB321 & Is there a large class of nouns whose gender/noun class is not phonologically or semantically predictable? \\ 
  GB322 & Is there grammatical marking of direct evidence (perceived with the senses)? \\ 
  GB323 & Is there grammatical marking of indirect evidence (hearsay, inference, etc.)? \\ 
  GB324 & Is there an interrogative verb for content interrogatives (who?, what?, etc.)? \\ 
  GB325 & Is there a count/mass distinction in interrogative quantifiers? \\ 
  GB326 & Do (nominal) content interrogatives normally or frequently occur in situ? \\ 
  GB327 & Can the relative clause follow the noun? \\ 
  GB328 & Can the relative clause precede the noun? \\ 
  GB329 & Are there internally-headed relative clauses? \\ 
  GB330 & Are there correlative relative clauses? \\ 
  GB331 & Are there non-adjacent relative clauses? \\ 
  GB333 & Is there a decimal numeral system? \\ 
  GB334 & Is there synchronic evidence for any element of a quinary numeral system? \\ 
  GB335 & Is there synchronic evidence for any element of a vigesimal numeral system? \\ 
  GB336 & Is there a body-part tallying system? \\ 
  GB400 & Are all person categories neutralized in some voice, tense, aspect, mood and/or negation? \\ 
  GB401 & Is there a class of patient-labile verbs? \\ 
  GB402 & Does the verb for 'see' have suppletive verb forms? \\ 
  GB403 & Does the verb for 'come' have suppletive verb forms? \\ 
  GB408 & Is there any accusative alignment of flagging? \\ 
  GB409 & Is there any ergative alignment of flagging? \\ 
  GB410 & Is there any neutral alignment of flagging? \\ 
  GB415 & Is there a politeness distinction in 2nd person forms? \\ 
  GB421 & Is there a preposed complementizer in complements of verbs of thinking and/or knowing? \\ 
  GB422 & Is there a postposed complementizer in complements of verbs of thinking and/or knowing? \\ 
  GB430 & Can adnominal possession be marked by a prefix on the possessor? \\ 
  GB431 & Can adnominal possession be marked by a prefix on the possessed noun? \\ 
  GB432 & Can adnominal possession be marked by a suffix on the possessor? \\ 
  GB433 & Can adnominal possession be marked by a suffix on the possessed noun? \\ 
  GB519 & Can mood be marked by a non-inflecting word (""""""""auxiliary particle"""""""")? \\ 
  GB520 & Can aspect be marked by a non-inflecting word (""""""""auxiliary particle"""""""")? \\ 
  GB521 & Can tense be marked by a non-inflecting word (""""""""auxiliary particle"""""""")? \\ 
  GB522 & Can the S or A argument be omitted from a pragmatically unmarked clause when the referent is inferrable from context (""""""""pro-drop"""""""" or """"""""null anaphora"""""""")? \\ 
   \bottomrule
\caption{Table of Grambank fetures} 
\label{GB_features_table}
\end{longtable}
